\documentclass[10pt,a4paper]{article}
\usepackage[latin1]{inputenc}
\usepackage{amsmath}
\usepackage{amsfonts}
\usepackage{amssymb}
\author{Z-One Team}
\title{Documento de Design}
\begin{document}

\begin{flushleft}
{\bf Universidade de Brasília - Faculdade Gama}\\
{\bf Disciplina:} Introdução aos Jogos Eletronicos\\
{\bf Professor:} Edson Alves\\
{\bf Alunos:}\\
Tiago Gomes Pereira	09/0134222\\
Matheus Fonseca	10/0054650\\
Charles Oliveira	09/0043006\\  
Pedro Zanini	10/47337\\
\end{flushleft}



\begin{center}
Ninja Siege (Cerco Ninja)
Documento de Design
\end{center}

\section{Introdução}
	Este documento aborda as características estruturais do jogo a ser produzido pela Z-One Team. Serão detalhados o ambiente do jogo, isto é, em que momento da historia ele se passa e os elementos fundamentais que definem o jogo. Dentre os elementos especificados estarão os inimigos, torres, mercado de jogo, mapas, etc.

\section{Premissa e Ambientação do Jogo}
	Século XV, Japão Feudal, os ninjas de todo mundo se reuniram para reunião secreta mais importante dos ninjas, a escolha do novo Mestre Supremo dos Ninjas. A reunião é no topo da Torre da Escuridão, uma torre de 12 andares que é o centro de comando de todos os ninjas.

	Mas os detalhes dessa reunião chegaram aos ouvidos dos maiores inimigos dos ninjas: os piratas e os samurais. Eles resolveram aproveitar essa oportunidade para fazer um cerco e atacar os ninjas enquanto estão dentro de sua torre e acabar de vez com as maiores lideranças ninjas.

    Você, como o novo mestre supremo dos ninjas, \textbf{Sr. Myagi}, deve guiar os seus ninjas até a base da torre se defendendo dos piratas e samurais que tem como objetivo invadir e matar o \textbf{Sr. Myagi}.

\section{Elementos do Jogo}
\subsection{Waves}
Cada andar possui um número de ondas contínuas de inimigos, podendo variar dependendo do andar. No início do jogo o jogador pode tomar o tempo que for necessário para gastar os recursos iniciais construindo suas torres. Apos construir suas torres e se sentir satisfeito ele inicia o jogo e as ondas de inimigos atacam continuamente com intervalos entre as ondas.
O jogador possuirá a opção de mandar a proxima onda mais cedo. E quando o jogo estiver pausado o jogador não poderá construir torres.
\subsection{Inimigos}
\subsubsection{Samurais}
\subsubsection{Piratas}

\subsection{Torres}
\subsubsection{Curto Alcance}
\subsubsection{Longo Alcance}
\subsubsection{Armadilhas}

\subsection{Percurso}
\subsection{Mapas}
\end{document}