\documentclass[10pt,a4paper]{article}

\usepackage[brazilian]{babel}
\usepackage[utf8]{inputenc}
\usepackage[T1]{fontenc}
\usepackage{graphicx}
\usepackage{amsmath}
\usepackage{amsfonts}
\usepackage{amssymb}

\pagestyle{empty}

\author{Z-One Team}
\title{Ninja Siege - Documento de Design}



\begin{document}

\maketitle 

\tableofcontents

\section{Introdução}
	Este documento aborda as características estruturais do jogo a ser produzido pela Z-One Team. Serão detalhados o ambiente do jogo, isto é, em que momento da historia ele se passa e os elementos fundamentais que definem o jogo. Dentre os elementos especificados estarão os inimigos, torres, mercado de jogo, mapas, etc.\\

\section{Premissa e Ambientação do Jogo}
	Século XV, Japão Feudal, os ninjas de todo mundo se reuniram para reunião secreta mais importante dos ninjas, a escolha do novo Mestre Supremo dos Ninjas. A reunião é no topo da Torre da Escuridão, uma torre de 12 andares que é o centro de comando de todos os ninjas.\\
	Mas os detalhes dessa reunião chegaram aos ouvidos dos maiores inimigos dos ninjas: os piratas e os samurais. Eles resolveram aproveitar essa oportunidade para fazer um cerco e atacar os ninjas enquanto estão dentro de sua torre e acabar de vez com as maiores lideranças ninjas.\\
    Você, como o novo mestre supremo dos ninjas, \textbf{Sr. Myagi}, deve guiar os seus ninjas até a base da torre se defendendo dos piratas e samurais que tem como objetivo invadir e matar o \textbf{Sr. Myagi}.\\

\section{Elementos do Jogo}

\subsection{Waves}
	Cada andar possui um número de ondas contínuas de inimigos, podendo variar dependendo do andar. No início do jogo o jogador pode tomar o tempo que for necessário para gastar os recursos iniciais construindo suas torres. Apos construir suas torres e se sentir satisfeito ele inicia o jogo e as ondas de inimigos atacam continuamente com intervalos entre as ondas.\\
	O jogador possuirá a opção de mandar a proxima onda mais cedo. E quando o jogo estiver pausado o jogador não poderá construir torres.\\

\subsection{Inimigos}
\subsubsection{Samurais}
\begin{itemize}
\item \textbf{Soldado (Ashigaru):} Samurai de classe baixa equipado com um par de facas.
	\begin{itemize}
	\item Velocidade: Média
	\item Vida: Baixa
	\item Resistência: Média
	\end{itemize}
\item \textbf{Retalhador (Batosai):} Samurai com sede de sangue, extremamente veloz e fatal, ficando equipado com duas katanas.
	\begin{itemize}
	\item Velocidade: Alta
	\item Vida: Baixa
	\item Resistência: Baixa
	\end{itemize}
\item \textbf{Mestre do Dojo:} Samurai experiente em batalhas equipado com um uma katana.
	\begin{itemize}
	\item Velocidade: Baixa
	\item Vida: Média
	\item Resistência: Alta
	\end{itemize}
\item \textbf{Espada-Santa:} Samurai que possui técnicas supremas, secretas e sobre-humanas.
	\begin{itemize}
	\item Velocidade: Média
	\item Vida: Alta
	\item Resistência: Média
	\end{itemize}
\end{itemize}

\subsubsection{Piratas}
\begin{itemize}
\item \textbf{Saqueadores:} Pirata comum equipado um mosquete (Estilo: Normal).
	\begin{itemize}
	\item Velocidade:
	\item Vida:
	\item Resistência:
	\end{itemize}
\item \textbf{Corsários:} Pirata saqueador de grandes embarcaçoes, equipado com um sabre (Estilo: Rápido).
	\begin{itemize}
	\item Velocidade:
	\item Vida:
	\item Resistência:
	\end{itemize}
\item \textbf{Perna de pau:} Pirata experiente com deficiência em uma perna, equipado com um punhal (Estilo: Resistente e lento).
	\begin{itemize}
	\item Velocidade:
	\item Vida:
	\item Resistência:
	\end{itemize}
\item \textbf{Capitão:} Pirata comandante, extremamente cruel e cauteloso, equipado com um gancho e um sabre (Estilo:Boss).
	\begin{itemize}
	\item Velocidade:
	\item Vida:
	\item Resistência:
	\end{itemize}
\end{itemize}

\subsection{Torres}
	As torres serão os Ninjas. Desde antigamente os ninjas são mortais inimigos de samurais e
piratas. Durante o jogo os Piratas e Samurais tentam chegar até o mestre supremo dos ninjas. Cada
ninja pode ser evoluido, aumentando seus atributos, como o dano causado e a velocidade de ataque.\\
\subsubsection{Curto Alcance}
\begin{itemize}
\item Katana Ninja
\item Nunchaku Ninja
\item Manriki Ninja
\end{itemize}

\subsubsection{Longo Alcance}
\begin{itemize}
\item Kunai Ninja
\item Shuriken Ninja
\item Ninja Jogador de Bombas
\item Ninja que joga facas
\end{itemize}

\subsubsection{Armadilhas}
\begin{itemize}
\item Armadilha de Urso
\item Parede com Espinhos
\end{itemize}

\subsection{Percurso}
	Cada mapa terá um caminho fixo que os inimigos irão percorrer. Cada mapa representa um
andar e o jogo terá no total 12 andares. Em cada mapa também será definido o número máximo de
torres que podem ser instaladas e o número máximo de ondas.\\
\subsubsection{Posicionamento das Torres}
	O posicionamento das torres será variável. O jogador poderá posiciona-las em qualquer lugar no
mapa em que se possa construir torres, ou seja, qualquer lugar que não seja o caminho ou obstáculos
(pilastras, mesas, outras torres, etc). Os lugares possíveis de posicionamento de torres serão específicos
de cada mapa.\\
\subsubsection{Relação Torre X Inimigo}
	Samurais e piratas não podem destruir ninjas, com exceção do ninja superior.\\

\subsection{Mapas}
	O jogo possuirá 13 mapas, sendo que 12 mapa representará um ambiente da torre (um andar) e
um ambiente externo, o jardim. Os ambientes serão:
\begin{enumerate}
\item Jardim externo
MATHEUS
\item Saguão de Entrada
TIAGO
\item Sala de Treinos (Dojo)
CHARLES
\item Sala de jantar
PEDRO
\item Sala de Armas
PEDRO
\item Sala de banhos
CHARLES
\item Biblioteca
MATHEUS
\item Sala das Artes Negras
TIAGO
\item Armazém (onde os mantimentos são guardados)
PEDRO
\item Dormitório
MATHEUS
\item Masmorras
TIAGO
\item Sala da Guarda
CHARLES
\item Sala de Reuniões
TIAGO

\end{enumerate}

\section{Mercado do jogo}
\subsection{Moeda}
	Para compra de torres e aquisição de melhorias das mesmas será necessária o gasto de pontos
de experiência "exp". A experiência é adquirida ao destruir uma unidade inimiga (varia em relação a
dificuldade da unidade inimiga).
\subsection{Evolução de unidades}
	Cada torre em jogo terá três níveis de evolução (o primeiro nível é o estado inicial). Cada
evolução terá um custo de experiência variável, onde quanto maior o nível, maior é o custo.

\section{Interfaces}
\subsection{Audio}
Haverá música de fundo para navegação de menus e enquanto não tiver acontecendo ataques
para cada fase. Para cada ataque das torres será tocado um som da arma usada. Ex: o som de uma lança
ao se usar uma Katana Ninja.
\subsection{Video}
O jogo terá uma visão over-head. As torres/inimigos no entanto serão vistas de lado. A câmera
será fixa, assim como o mapa do jogo, durante todas as etapas do jogo. Terá o tamanho de 800x600
pixels, ou seja, o tamanho do mapa é no máximo do tamanho da tela.
\subsection{Controles}
Haverá o posicionamento das torres através da movimentação do mouse. O jogador mais
experiênte pode usufruir das teclas de atalho para selecionar uma torre e coloca-la no mapa apertando
letras do teclado, assim a torre pode ser posicionada com o auxílio do mouse.

\section{Plataforma}
	O Ninja Seige rodará apenas em ambiente Linux.

\section{Audiência}
	O jogo foca pessoas de idade entre 10 a 30 anos de idade não importando o gênero. Jogadores
com qualquer renda, que dispuserem de um computador, poderão se divertir com o Ninja Seige. O jogo
também é voltado para o perfil mais casual, pois tem o objetivo de entreter a pessoa com um
passatempo divertido.

\section{Desenvolvimento}
\subsection{Linguagem compilada}
	Será utilizado o C++ como linguagem, pois contempla todos os recursos de baixo nível da
linguagem C, como gerenciamento manual de memória, manipulação de ponteiros e linguagem próxima
do nível de máquina, ao mesmo tempo em que contempla recursos de linguagem de alto nível, como
orientação a objetos e uso de namespaces. Em concomitância com o C++ será utilizada a biblioteca SDL
1.2.15 para Linux. O ambiente de desenvolvimento será Ubuntu. O compilador a ser utilizado será o g++.

\subsection{Linguagem de script}
	Será utilizado o Python 2.7 como linguagem de script pelo domínio que os integrantes de
programação já possuem da linguagem e por já ser nativa de sistemas Linux. Python é uma linguagem
multiparadigmas que permite programação estruturada, orientada a objetos e ainda permite programar
com paradigma funcional, este que é visto em linguagens de nível bastante alto. Como é uma linguagem
interpretada, questões que poderão sofrer grandes alterações durante o desenvolvimento (ex:
Inteligência Artificial) serão desenvolvidas na linguagem Python, eliminando o tempo de re-compilação
em cada modificação.

\subsection{Documentação}
	O Doxygen é uma ferramenta para documentação de código bastante completa e pode ser
usada em C, C++, java, Python entre outras. Ele permite a criação de páginas html que documentam
todo o código de uma forma elegante e completa como referência cruzada entre entidades e estruturas
do código como também mostra partes do código em si. O doxygen ainda permite a geração de páginas
man e documentos latex.

\subsection{Compartilhamento de código}
	O SVN é uma ferramenta para desenvolvimento de código em equipe utilizada para manter
versões e builds do código em um único lugar. A utilização do SVN evita repetição e perda de
versionamento de código, aspecto fundamental em projetos em grupo <http://code.google.com/p/z-
one/>.

\subsection{Editor de Texto}
	O editor de texto a ser utilizado será o gedit, no ambiente Linux, com os seguintes plugins:
\begin{itemize}
\item Autocomplete
\item Class Browser
\item Code Comment
\item Gemini
\item Snippets
\item Terminal
\end{itemize}
	A utilização dos plugins tem como objetivo auxiliar na tarefa de programação trazendo diversas
funções de grande utilidade na hora de programar.

\section{Anexos}
\subsection{Mapas}


\end{document}